\documentclass[11pt,aspectratio=169,t]{beamer}
%[t] forces slides to top align by default

\usepackage[T1]{fontenc}
%\setbeamertemplate{title page}{
%\noindent\includegraphics[width=\paperwidth,height=\paperheight]{title.png}
%}
\usetheme[sectionpage=simple]{ukceh}
\setbackgroundbaseimage{assets/tree}

%\setbeamertemplate{page number in head/foot}[appendixframenumber]
%\setbeamertemplate{section in toc}[sections numbered]

%\usepackage{booktabs}
%\usepackage[scale=2]{ccicons}

%\usepackage{xspace}

%\newcommand{\themename}{\textbf{ukceh}\xspace}

%\title{Moloch}
\title{Presentation title that can break over three lines if required}
\subtitle{Presentation subtitle goes here over two lines if required}
\date{25/08/29}
\author{Robin Long}
\institute{Digital Research Group, NCDR}
%\titlegraphic{\hfill\includegraphics[width=3.5cm]{ukceh-logo.pdf}}

\email{eds\_enquires@ceh.ac.uk}
\website{ceh.ac.uk}
\handle{@UK\_CEH}

\begin{document}

\maketitle

\begin{frame}
  \frametitle{Notes to the template:\\Type levels and bullets}

  The standard text area has five predefined type levels.
    
  You can move up or down the levels using the increase or decrease indent buttons in the home tab.
  
  We recommend that you do not use the bullet button, as this does not apply the correct bullet formatting.
  
  Level one is for general text
  
  Level two is for subheadings within body copy
  
\begin{itemize}
\item Level three is for intro text
\item Level four is the first level of bullets
  \begin{itemize}
  \item Level five is the second level of bullets
  \end{itemize}
\end{itemize}
\end{frame}

\begin{frame}
  \frametitle{Notes to the template:\\Type levels and bullets}

  The standard text area has five predefined type levels.
    
  You can move up or down the levels using the increase or decrease indent buttons in the home tab.
  
  We recommend that you do not use the bullet button, as this does not apply the correct bullet formatting.
  
  Level one is for general text
  
  Level two is for subheadings within body copy
  
\begin{itemize}
\item Level three is for intro text
\item Level four is the first level of bullets
  \begin{itemize}
  \item Level five is the second level of bullets
  \end{itemize}
\end{itemize}
\end{frame}





\section{Digital Innovation}


\begin{frame}
  \frametitle{Notes to the template:\\Type levels and bullets}

  The standard text area has five predefined type levels.
    
  You can move up or down the levels using the increase or decrease indent buttons in the home tab.
  
  We recommend that you do not use the bullet button, as this does not apply the correct bullet formatting.
  
  Level one is for general text
  
  Level two is for subheadings within body copy
  
\begin{itemize}
\item Level three is for intro text
\item Level four is the first level of bullets
  \begin{itemize}
  \item Level five is the second level of bullets
  \end{itemize}
\end{itemize}
\end{frame}

\begin{frame}
  \frametitle{Notes to the template:\\Type levels and bullets}

  The standard text area has five predefined type levels.
    
  You can move up or down the levels using the increase or decrease indent buttons in the home tab.
  
  We recommend that you do not use the bullet button, as this does not apply the correct bullet formatting.
  
  Level one is for general text
  
  Level two is for subheadings within body copy
  
\begin{itemize}
\item Level three is for intro text
\item Level four is the first level of bullets
  \begin{itemize}
  \item Level five is the second level of bullets
  \end{itemize}
\end{itemize}
\end{frame}

\begin{frame}
  \frametitle{Notes to the template:\\Type levels and bullets}

  The standard text area has five predefined type levels.
    
  You can move up or down the levels using the increase or decrease indent buttons in the home tab.
  
  We recommend that you do not use the bullet button, as this does not apply the correct bullet formatting.
  
  Level one is for general text
  
  Level two is for subheadings within body copy
  
\begin{itemize}
\item Level three is for intro text
\item Level four is the first level of bullets
  \begin{itemize}
  \item Level five is the second level of bullets
  \end{itemize}
\end{itemize}
\end{frame}

\begin{frame}
  \frametitle{Notes to the template:\\Type levels and bullets}

  The standard text area has five predefined type levels.
    
  You can move up or down the levels using the increase or decrease indent buttons in the home tab.
  
  We recommend that you do not use the bullet button, as this does not apply the correct bullet formatting.
  
  Level one is for general text
  
  Level two is for subheadings within body copy
  
\begin{itemize}
\item Level three is for intro text
\item Level four is the first level of bullets
  \begin{itemize}
  \item Level five is the second level of bullets
  \end{itemize}
\end{itemize}
\end{frame}

\end{document}
